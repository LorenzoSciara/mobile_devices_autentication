\chapter{Conclusions}
The goal of this project is to define a methodology to prove to some external verifier that those data were generated on that specific device. We started by researching what could be some ways to uniquely identify the device. Reading the android documentation we found several unique codes that could be right for us because some of them were even directly related to the hardware, such as the IMEI, the serial number of the device or the MAC address. However, we realized that the Android documentation discouraged the use of these codes for identification purposes since it is not a good practice to use non-modifiable codes for these purposes. Furthermore, starting from the Android 11 version onwards (API level $>$= 11) these codes are no longer accessible by third-party applications since a particular permission is required which is granted only to system apps carrier apps (ISP applications).
We also explored other types of codes (Android ID, Advertising ID) but they didn't fit as they weren't persistent (resettable after a factory reset) or changed after disinstalling and reinstalling the application.

Therefore, we opted to study aspects related to cryptographic operations of the device, such as the generation of keys, their signature and the digital certificates associated with them. We also discovered that it was possible to rely on the key attestation mechanism to verify the authenticity of key, which can also be hardware-backed thanks to the use of device security modules such as the \textbf{TEE (Trusted Execution Environment)} and the\textbf{ Android Keystore Module}.

So we decided to leverage these tools to find a way to uniquely bind the key (key-pair) generated and used to sign a sample data to the device that generated it (which generated the data and therefore also the key-pair).

We have exploited the fact that when a key-pair is generated on an Android device, it is signed (certified) by the TEE in order to guarantee its authenticity. In turn, the TEE is certified in a chain by other entities up to Google which is a publicly trusted Certification Authority. This certificate chain, together with the digital signature of the data, can be sent to the verifier. During the verification phase, it is possible to extract from the first certificate of the chain information on the issuer, i.e. on the entity who certified the key-pair, which would be the TEE of the sender device. By extracting this serial number, during the verification phase, it is then possible to compare it with the actual TEE serial number of the sender device that the verifier already knows in some way. In this way it is possible to uniquely identify the sender and then successfully bind the data to the device that generated and sent it.




\chapter{Introduction}
% DELETE THE TEXT BELOW
% In this first chapter we expect you to introduce the project explaining what the project is about, what is the final goal, what are the topics tackled by the project, etc.\newline The introduction must not include any low-level detail about the project, avoid sentences written like: we did this, then this, then this, etc.\newline It is strongly suggested to avoid expressions like `We think`, `We did`, etc...it is better to use impersonal expressions such as: `It is clear that`, `It is possible that`, `... something ... has been implemented/analyzed/etc.` (instead of `we did, we implemented, we analyzed`).\newline In the introduction you should give to the reader enough information to understand what is going to be explained in the remainder of the report (basically, expanding some concept you mentioned in the Abstract) without giving away too many information that would make the introduction too long and boring.\newline Feel free to organize the introduction in multiple sections and subsections, depending on how much content you want to put into this chapter.

% Remember that the introduction is needed to make the reader understand what kind of reading he or she will encounter. Be fluent and try not to confuse him or her.
% The introduction must ALWAYS end with the following formula: The remainder of the document is organized as follows. In Chapter 2, ...; in Chapter 3, ... so that the reader can choose which chapters are worth skipping according to the type of reading he or she has chosen.

The project aims to address the challenge of authenticating data originating from a smartphone device, by leveraging unique features specific to that device. The problem can be described as follows: Suppose there is a device, on which you have no physical control however, you can execute software that sends you data, which are purportedly generated on the device, such as a screenshot. The question then arises: How can you prove that the data indeed originated from the device and were not generated by a malicious third party?

The goal of this project is to define a methodology to prove to an external verifier that the data in question was indeed generated on the specific device. We have achieved this objective on Android smartphones leveraging the TEE (Trusted Execution Environment) of the device as Hardware-Backed root of trust for the correct usage of the Keystore module and the associated APIs provided by Android.
\\In practice, our software is able to generate an Asymmetric key-pair to perform some cryptographic operations, retrieve the associated certificate chain with which the key-pair is hierarchically signed, and perform the \textbf{Key Attestation operation} to attest the \textit{trustworthiness} of the certificates up to the root of trust represented by the Google certificate. This process allows us to obtain a \textbf{trusty and unique identifier} associated with the device, which is the TEE Serial Number of the attested device, that can efficiently guarantee and prove the authenticity of the sent data.
\\The project's approach involves several steps:
\begin{itemize}
    \item Firstly, we conducted a research to identify all possible identifiers present on an Android device, classifying them according to two main criteria: accessibility and uniqueness of the identifier.
    \item Unfortunately, the most interesting identifiers have proved impossible to use due to numerous limitations introduced in the latest versions of Android 
    \item Ultimately, we concluded that the most effective way to identify the device was through its serial number of the Trusted Execution Environment (TEE). The correctness of the TEE serial number is given by the verification of the certificate chain provided by the keystore.
\end{itemize}
An alternative approach to achieving our objective was to design software capable of retrieving various information related to the device's SIM card (or SIM cards in the case of multiple SIMs present in the same device), including the ICCID (Integrated Circuit Card ID). Although these pieces of information do not directly identify the device, they can be used as identifiers for the SIM card itself, thus potentially increasing the accuracy of identification when multiple identifiers are available.
\\The most effective way to proceed through each phase was to follow the documentation available on the official Android web page and simultaneously hands-on experimentation with each new feature using Android Studio.
% MANCA % The introduction must ALWAYS end with the following formula: The remainder of the document is organized as follows. In Chapter 2, ...; in Chapter 3, ... so that the reader can choose which chapters are worth skipping according to the type of reading he or she has chosen.
